
\documentclass[10pt]{article}
\usepackage{geometry}
\geometry{a4paper}

\usepackage{graphicx}

\usepackage[parfill]{parskip}

\usepackage{amsthm}
\usepackage[cm]{fullpage}
\usepackage{hyperref}
\usepackage{enumitem}
\usepackage{xcolor}

\usepackage{listings}
\def\ContinueLineNumber{\lstset{firstnumber=last}}
\def\StartLineAt#1{\lstset{firstnumber=#1}}
\let\numberLineAt\StartLineAt

\lstset{
  basicstyle=\ttfamily,
  breaklines=true,
  frame=none,
  keepspaces=true,
  numbers=left,
  numbersep=5pt,
  numberstyle=\tiny\color{black!80},
  xleftmargin=.153in,
  language=Prolog,
  deletekeywords={clause,not,var},
  commentstyle=\color{black!40},
  escapechar=|,
  aboveskip=\smallskipamount,
  belowskip=\smallskipamount,
  backgroundcolor=\color{black!10},
}

\theoremstyle{definition}
\newtheorem{exercise}{Exercise}
\newtheorem{definition}{Definition}
\newtheorem{remark}{Remark}

%%% USE THIS TO SHOW SOLUTIONS
\newcommand{\innersolutions}[1]{#1}
%%% USE THIS TO HIDE SOLUTIONS
%\newcommand{\innersolutions}[1]{}
%%%


\newcommand{\separator}[0]{
  \textcolor{black!80}{\rule{\textwidth}{1.5pt}}
}

\title{\sffamily \bfseries Knowledge Representation and Reasoning 2022\\
\sffamily \bfseries {\Large Homework assignment \#$x$}}
\author{Firstname Lastname (student number 0000000), Firstname Lastname (student number 0000000)}
\date{}

\begin{document}
\maketitle

\begin{exercise}
Here you can write your solution for Exercise~1.

Perhaps you might want to mention an answer set program:

\begin{lstlisting}
a :- not b.
b :- not a.
c :- not d.
d :- not c.
\end{lstlisting}

And another one:

\ContinueLineNumber
\begin{lstlisting}
e :- not f.
f :- not e.
\end{lstlisting}
\end{exercise}

\separator{}

\begin{exercise}
Here you can write your solution for another exercise.
\begin{enumerate}[label=(\alph*)]
  \item This is the first part.
  \item And this is the second part.
\end{enumerate}
\end{exercise}

\end{document}  
